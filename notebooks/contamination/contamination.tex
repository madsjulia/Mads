\documentclass[11pt]{article}

    \usepackage[breakable]{tcolorbox}
    \usepackage{parskip} % Stop auto-indenting (to mimic markdown behaviour)
    
    \usepackage{iftex}
    \ifPDFTeX
    	\usepackage[T1]{fontenc}
    	\usepackage{mathpazo}
    \else
    	\usepackage{fontspec}
    \fi

    % Basic figure setup, for now with no caption control since it's done
    % automatically by Pandoc (which extracts ![](path) syntax from Markdown).
    \usepackage{graphicx}
    % Maintain compatibility with old templates. Remove in nbconvert 6.0
    \let\Oldincludegraphics\includegraphics
    % Ensure that by default, figures have no caption (until we provide a
    % proper Figure object with a Caption API and a way to capture that
    % in the conversion process - todo).
    \usepackage{caption}
    \DeclareCaptionFormat{nocaption}{}
    \captionsetup{format=nocaption,aboveskip=0pt,belowskip=0pt}

    \usepackage{float}
    \floatplacement{figure}{H} % forces figures to be placed at the correct location
    \usepackage{xcolor} % Allow colors to be defined
    \usepackage{enumerate} % Needed for markdown enumerations to work
    \usepackage{geometry} % Used to adjust the document margins
    \usepackage{amsmath} % Equations
    \usepackage{amssymb} % Equations
    \usepackage{textcomp} % defines textquotesingle
    % Hack from http://tex.stackexchange.com/a/47451/13684:
    \AtBeginDocument{%
        \def\PYZsq{\textquotesingle}% Upright quotes in Pygmentized code
    }
    \usepackage{upquote} % Upright quotes for verbatim code
    \usepackage{eurosym} % defines \euro
    \usepackage[mathletters]{ucs} % Extended unicode (utf-8) support
    \usepackage{fancyvrb} % verbatim replacement that allows latex
    \usepackage{grffile} % extends the file name processing of package graphics 
                         % to support a larger range
    \makeatletter % fix for old versions of grffile with XeLaTeX
    \@ifpackagelater{grffile}{2019/11/01}
    {
      % Do nothing on new versions
    }
    {
      \def\Gread@@xetex#1{%
        \IfFileExists{"\Gin@base".bb}%
        {\Gread@eps{\Gin@base.bb}}%
        {\Gread@@xetex@aux#1}%
      }
    }
    \makeatother
    \usepackage[Export]{adjustbox} % Used to constrain images to a maximum size
    \adjustboxset{max size={0.9\linewidth}{0.9\paperheight}}

    % The hyperref package gives us a pdf with properly built
    % internal navigation ('pdf bookmarks' for the table of contents,
    % internal cross-reference links, web links for URLs, etc.)
    \usepackage{hyperref}
    % The default LaTeX title has an obnoxious amount of whitespace. By default,
    % titling removes some of it. It also provides customization options.
    \usepackage{titling}
    \usepackage{longtable} % longtable support required by pandoc >1.10
    \usepackage{booktabs}  % table support for pandoc > 1.12.2
    \usepackage[inline]{enumitem} % IRkernel/repr support (it uses the enumerate* environment)
    \usepackage[normalem]{ulem} % ulem is needed to support strikethroughs (\sout)
                                % normalem makes italics be italics, not underlines
    \usepackage{mathrsfs}
    

    
    % Colors for the hyperref package
    \definecolor{urlcolor}{rgb}{0,.145,.698}
    \definecolor{linkcolor}{rgb}{.71,0.21,0.01}
    \definecolor{citecolor}{rgb}{.12,.54,.11}

    % ANSI colors
    \definecolor{ansi-black}{HTML}{3E424D}
    \definecolor{ansi-black-intense}{HTML}{282C36}
    \definecolor{ansi-red}{HTML}{E75C58}
    \definecolor{ansi-red-intense}{HTML}{B22B31}
    \definecolor{ansi-green}{HTML}{00A250}
    \definecolor{ansi-green-intense}{HTML}{007427}
    \definecolor{ansi-yellow}{HTML}{DDB62B}
    \definecolor{ansi-yellow-intense}{HTML}{B27D12}
    \definecolor{ansi-blue}{HTML}{208FFB}
    \definecolor{ansi-blue-intense}{HTML}{0065CA}
    \definecolor{ansi-magenta}{HTML}{D160C4}
    \definecolor{ansi-magenta-intense}{HTML}{A03196}
    \definecolor{ansi-cyan}{HTML}{60C6C8}
    \definecolor{ansi-cyan-intense}{HTML}{258F8F}
    \definecolor{ansi-white}{HTML}{C5C1B4}
    \definecolor{ansi-white-intense}{HTML}{A1A6B2}
    \definecolor{ansi-default-inverse-fg}{HTML}{FFFFFF}
    \definecolor{ansi-default-inverse-bg}{HTML}{000000}

    % common color for the border for error outputs.
    \definecolor{outerrorbackground}{HTML}{FFDFDF}

    % commands and environments needed by pandoc snippets
    % extracted from the output of `pandoc -s`
    \providecommand{\tightlist}{%
      \setlength{\itemsep}{0pt}\setlength{\parskip}{0pt}}
    \DefineVerbatimEnvironment{Highlighting}{Verbatim}{commandchars=\\\{\}}
    % Add ',fontsize=\small' for more characters per line
    \newenvironment{Shaded}{}{}
    \newcommand{\KeywordTok}[1]{\textcolor[rgb]{0.00,0.44,0.13}{\textbf{{#1}}}}
    \newcommand{\DataTypeTok}[1]{\textcolor[rgb]{0.56,0.13,0.00}{{#1}}}
    \newcommand{\DecValTok}[1]{\textcolor[rgb]{0.25,0.63,0.44}{{#1}}}
    \newcommand{\BaseNTok}[1]{\textcolor[rgb]{0.25,0.63,0.44}{{#1}}}
    \newcommand{\FloatTok}[1]{\textcolor[rgb]{0.25,0.63,0.44}{{#1}}}
    \newcommand{\CharTok}[1]{\textcolor[rgb]{0.25,0.44,0.63}{{#1}}}
    \newcommand{\StringTok}[1]{\textcolor[rgb]{0.25,0.44,0.63}{{#1}}}
    \newcommand{\CommentTok}[1]{\textcolor[rgb]{0.38,0.63,0.69}{\textit{{#1}}}}
    \newcommand{\OtherTok}[1]{\textcolor[rgb]{0.00,0.44,0.13}{{#1}}}
    \newcommand{\AlertTok}[1]{\textcolor[rgb]{1.00,0.00,0.00}{\textbf{{#1}}}}
    \newcommand{\FunctionTok}[1]{\textcolor[rgb]{0.02,0.16,0.49}{{#1}}}
    \newcommand{\RegionMarkerTok}[1]{{#1}}
    \newcommand{\ErrorTok}[1]{\textcolor[rgb]{1.00,0.00,0.00}{\textbf{{#1}}}}
    \newcommand{\NormalTok}[1]{{#1}}
    
    % Additional commands for more recent versions of Pandoc
    \newcommand{\ConstantTok}[1]{\textcolor[rgb]{0.53,0.00,0.00}{{#1}}}
    \newcommand{\SpecialCharTok}[1]{\textcolor[rgb]{0.25,0.44,0.63}{{#1}}}
    \newcommand{\VerbatimStringTok}[1]{\textcolor[rgb]{0.25,0.44,0.63}{{#1}}}
    \newcommand{\SpecialStringTok}[1]{\textcolor[rgb]{0.73,0.40,0.53}{{#1}}}
    \newcommand{\ImportTok}[1]{{#1}}
    \newcommand{\DocumentationTok}[1]{\textcolor[rgb]{0.73,0.13,0.13}{\textit{{#1}}}}
    \newcommand{\AnnotationTok}[1]{\textcolor[rgb]{0.38,0.63,0.69}{\textbf{\textit{{#1}}}}}
    \newcommand{\CommentVarTok}[1]{\textcolor[rgb]{0.38,0.63,0.69}{\textbf{\textit{{#1}}}}}
    \newcommand{\VariableTok}[1]{\textcolor[rgb]{0.10,0.09,0.49}{{#1}}}
    \newcommand{\ControlFlowTok}[1]{\textcolor[rgb]{0.00,0.44,0.13}{\textbf{{#1}}}}
    \newcommand{\OperatorTok}[1]{\textcolor[rgb]{0.40,0.40,0.40}{{#1}}}
    \newcommand{\BuiltInTok}[1]{{#1}}
    \newcommand{\ExtensionTok}[1]{{#1}}
    \newcommand{\PreprocessorTok}[1]{\textcolor[rgb]{0.74,0.48,0.00}{{#1}}}
    \newcommand{\AttributeTok}[1]{\textcolor[rgb]{0.49,0.56,0.16}{{#1}}}
    \newcommand{\InformationTok}[1]{\textcolor[rgb]{0.38,0.63,0.69}{\textbf{\textit{{#1}}}}}
    \newcommand{\WarningTok}[1]{\textcolor[rgb]{0.38,0.63,0.69}{\textbf{\textit{{#1}}}}}
    
    
    % Define a nice break command that doesn't care if a line doesn't already
    % exist.
    \def\br{\hspace*{\fill} \\* }
    % Math Jax compatibility definitions
    \def\gt{>}
    \def\lt{<}
    \let\Oldtex\TeX
    \let\Oldlatex\LaTeX
    \renewcommand{\TeX}{\textrm{\Oldtex}}
    \renewcommand{\LaTeX}{\textrm{\Oldlatex}}
    % Document parameters
    % Document title
    \title{contamination}
    
    
    
    
    
% Pygments definitions
\makeatletter
\def\PY@reset{\let\PY@it=\relax \let\PY@bf=\relax%
    \let\PY@ul=\relax \let\PY@tc=\relax%
    \let\PY@bc=\relax \let\PY@ff=\relax}
\def\PY@tok#1{\csname PY@tok@#1\endcsname}
\def\PY@toks#1+{\ifx\relax#1\empty\else%
    \PY@tok{#1}\expandafter\PY@toks\fi}
\def\PY@do#1{\PY@bc{\PY@tc{\PY@ul{%
    \PY@it{\PY@bf{\PY@ff{#1}}}}}}}
\def\PY#1#2{\PY@reset\PY@toks#1+\relax+\PY@do{#2}}

\@namedef{PY@tok@w}{\def\PY@tc##1{\textcolor[rgb]{0.73,0.73,0.73}{##1}}}
\@namedef{PY@tok@c}{\let\PY@it=\textit\def\PY@tc##1{\textcolor[rgb]{0.25,0.50,0.50}{##1}}}
\@namedef{PY@tok@cp}{\def\PY@tc##1{\textcolor[rgb]{0.74,0.48,0.00}{##1}}}
\@namedef{PY@tok@k}{\let\PY@bf=\textbf\def\PY@tc##1{\textcolor[rgb]{0.00,0.50,0.00}{##1}}}
\@namedef{PY@tok@kp}{\def\PY@tc##1{\textcolor[rgb]{0.00,0.50,0.00}{##1}}}
\@namedef{PY@tok@kt}{\def\PY@tc##1{\textcolor[rgb]{0.69,0.00,0.25}{##1}}}
\@namedef{PY@tok@o}{\def\PY@tc##1{\textcolor[rgb]{0.40,0.40,0.40}{##1}}}
\@namedef{PY@tok@ow}{\let\PY@bf=\textbf\def\PY@tc##1{\textcolor[rgb]{0.67,0.13,1.00}{##1}}}
\@namedef{PY@tok@nb}{\def\PY@tc##1{\textcolor[rgb]{0.00,0.50,0.00}{##1}}}
\@namedef{PY@tok@nf}{\def\PY@tc##1{\textcolor[rgb]{0.00,0.00,1.00}{##1}}}
\@namedef{PY@tok@nc}{\let\PY@bf=\textbf\def\PY@tc##1{\textcolor[rgb]{0.00,0.00,1.00}{##1}}}
\@namedef{PY@tok@nn}{\let\PY@bf=\textbf\def\PY@tc##1{\textcolor[rgb]{0.00,0.00,1.00}{##1}}}
\@namedef{PY@tok@ne}{\let\PY@bf=\textbf\def\PY@tc##1{\textcolor[rgb]{0.82,0.25,0.23}{##1}}}
\@namedef{PY@tok@nv}{\def\PY@tc##1{\textcolor[rgb]{0.10,0.09,0.49}{##1}}}
\@namedef{PY@tok@no}{\def\PY@tc##1{\textcolor[rgb]{0.53,0.00,0.00}{##1}}}
\@namedef{PY@tok@nl}{\def\PY@tc##1{\textcolor[rgb]{0.63,0.63,0.00}{##1}}}
\@namedef{PY@tok@ni}{\let\PY@bf=\textbf\def\PY@tc##1{\textcolor[rgb]{0.60,0.60,0.60}{##1}}}
\@namedef{PY@tok@na}{\def\PY@tc##1{\textcolor[rgb]{0.49,0.56,0.16}{##1}}}
\@namedef{PY@tok@nt}{\let\PY@bf=\textbf\def\PY@tc##1{\textcolor[rgb]{0.00,0.50,0.00}{##1}}}
\@namedef{PY@tok@nd}{\def\PY@tc##1{\textcolor[rgb]{0.67,0.13,1.00}{##1}}}
\@namedef{PY@tok@s}{\def\PY@tc##1{\textcolor[rgb]{0.73,0.13,0.13}{##1}}}
\@namedef{PY@tok@sd}{\let\PY@it=\textit\def\PY@tc##1{\textcolor[rgb]{0.73,0.13,0.13}{##1}}}
\@namedef{PY@tok@si}{\let\PY@bf=\textbf\def\PY@tc##1{\textcolor[rgb]{0.73,0.40,0.53}{##1}}}
\@namedef{PY@tok@se}{\let\PY@bf=\textbf\def\PY@tc##1{\textcolor[rgb]{0.73,0.40,0.13}{##1}}}
\@namedef{PY@tok@sr}{\def\PY@tc##1{\textcolor[rgb]{0.73,0.40,0.53}{##1}}}
\@namedef{PY@tok@ss}{\def\PY@tc##1{\textcolor[rgb]{0.10,0.09,0.49}{##1}}}
\@namedef{PY@tok@sx}{\def\PY@tc##1{\textcolor[rgb]{0.00,0.50,0.00}{##1}}}
\@namedef{PY@tok@m}{\def\PY@tc##1{\textcolor[rgb]{0.40,0.40,0.40}{##1}}}
\@namedef{PY@tok@gh}{\let\PY@bf=\textbf\def\PY@tc##1{\textcolor[rgb]{0.00,0.00,0.50}{##1}}}
\@namedef{PY@tok@gu}{\let\PY@bf=\textbf\def\PY@tc##1{\textcolor[rgb]{0.50,0.00,0.50}{##1}}}
\@namedef{PY@tok@gd}{\def\PY@tc##1{\textcolor[rgb]{0.63,0.00,0.00}{##1}}}
\@namedef{PY@tok@gi}{\def\PY@tc##1{\textcolor[rgb]{0.00,0.63,0.00}{##1}}}
\@namedef{PY@tok@gr}{\def\PY@tc##1{\textcolor[rgb]{1.00,0.00,0.00}{##1}}}
\@namedef{PY@tok@ge}{\let\PY@it=\textit}
\@namedef{PY@tok@gs}{\let\PY@bf=\textbf}
\@namedef{PY@tok@gp}{\let\PY@bf=\textbf\def\PY@tc##1{\textcolor[rgb]{0.00,0.00,0.50}{##1}}}
\@namedef{PY@tok@go}{\def\PY@tc##1{\textcolor[rgb]{0.53,0.53,0.53}{##1}}}
\@namedef{PY@tok@gt}{\def\PY@tc##1{\textcolor[rgb]{0.00,0.27,0.87}{##1}}}
\@namedef{PY@tok@err}{\def\PY@bc##1{{\setlength{\fboxsep}{\string -\fboxrule}\fcolorbox[rgb]{1.00,0.00,0.00}{1,1,1}{\strut ##1}}}}
\@namedef{PY@tok@kc}{\let\PY@bf=\textbf\def\PY@tc##1{\textcolor[rgb]{0.00,0.50,0.00}{##1}}}
\@namedef{PY@tok@kd}{\let\PY@bf=\textbf\def\PY@tc##1{\textcolor[rgb]{0.00,0.50,0.00}{##1}}}
\@namedef{PY@tok@kn}{\let\PY@bf=\textbf\def\PY@tc##1{\textcolor[rgb]{0.00,0.50,0.00}{##1}}}
\@namedef{PY@tok@kr}{\let\PY@bf=\textbf\def\PY@tc##1{\textcolor[rgb]{0.00,0.50,0.00}{##1}}}
\@namedef{PY@tok@bp}{\def\PY@tc##1{\textcolor[rgb]{0.00,0.50,0.00}{##1}}}
\@namedef{PY@tok@fm}{\def\PY@tc##1{\textcolor[rgb]{0.00,0.00,1.00}{##1}}}
\@namedef{PY@tok@vc}{\def\PY@tc##1{\textcolor[rgb]{0.10,0.09,0.49}{##1}}}
\@namedef{PY@tok@vg}{\def\PY@tc##1{\textcolor[rgb]{0.10,0.09,0.49}{##1}}}
\@namedef{PY@tok@vi}{\def\PY@tc##1{\textcolor[rgb]{0.10,0.09,0.49}{##1}}}
\@namedef{PY@tok@vm}{\def\PY@tc##1{\textcolor[rgb]{0.10,0.09,0.49}{##1}}}
\@namedef{PY@tok@sa}{\def\PY@tc##1{\textcolor[rgb]{0.73,0.13,0.13}{##1}}}
\@namedef{PY@tok@sb}{\def\PY@tc##1{\textcolor[rgb]{0.73,0.13,0.13}{##1}}}
\@namedef{PY@tok@sc}{\def\PY@tc##1{\textcolor[rgb]{0.73,0.13,0.13}{##1}}}
\@namedef{PY@tok@dl}{\def\PY@tc##1{\textcolor[rgb]{0.73,0.13,0.13}{##1}}}
\@namedef{PY@tok@s2}{\def\PY@tc##1{\textcolor[rgb]{0.73,0.13,0.13}{##1}}}
\@namedef{PY@tok@sh}{\def\PY@tc##1{\textcolor[rgb]{0.73,0.13,0.13}{##1}}}
\@namedef{PY@tok@s1}{\def\PY@tc##1{\textcolor[rgb]{0.73,0.13,0.13}{##1}}}
\@namedef{PY@tok@mb}{\def\PY@tc##1{\textcolor[rgb]{0.40,0.40,0.40}{##1}}}
\@namedef{PY@tok@mf}{\def\PY@tc##1{\textcolor[rgb]{0.40,0.40,0.40}{##1}}}
\@namedef{PY@tok@mh}{\def\PY@tc##1{\textcolor[rgb]{0.40,0.40,0.40}{##1}}}
\@namedef{PY@tok@mi}{\def\PY@tc##1{\textcolor[rgb]{0.40,0.40,0.40}{##1}}}
\@namedef{PY@tok@il}{\def\PY@tc##1{\textcolor[rgb]{0.40,0.40,0.40}{##1}}}
\@namedef{PY@tok@mo}{\def\PY@tc##1{\textcolor[rgb]{0.40,0.40,0.40}{##1}}}
\@namedef{PY@tok@ch}{\let\PY@it=\textit\def\PY@tc##1{\textcolor[rgb]{0.25,0.50,0.50}{##1}}}
\@namedef{PY@tok@cm}{\let\PY@it=\textit\def\PY@tc##1{\textcolor[rgb]{0.25,0.50,0.50}{##1}}}
\@namedef{PY@tok@cpf}{\let\PY@it=\textit\def\PY@tc##1{\textcolor[rgb]{0.25,0.50,0.50}{##1}}}
\@namedef{PY@tok@c1}{\let\PY@it=\textit\def\PY@tc##1{\textcolor[rgb]{0.25,0.50,0.50}{##1}}}
\@namedef{PY@tok@cs}{\let\PY@it=\textit\def\PY@tc##1{\textcolor[rgb]{0.25,0.50,0.50}{##1}}}

\def\PYZbs{\char`\\}
\def\PYZus{\char`\_}
\def\PYZob{\char`\{}
\def\PYZcb{\char`\}}
\def\PYZca{\char`\^}
\def\PYZam{\char`\&}
\def\PYZlt{\char`\<}
\def\PYZgt{\char`\>}
\def\PYZsh{\char`\#}
\def\PYZpc{\char`\%}
\def\PYZdl{\char`\$}
\def\PYZhy{\char`\-}
\def\PYZsq{\char`\'}
\def\PYZdq{\char`\"}
\def\PYZti{\char`\~}
% for compatibility with earlier versions
\def\PYZat{@}
\def\PYZlb{[}
\def\PYZrb{]}
\makeatother


    % For linebreaks inside Verbatim environment from package fancyvrb. 
    \makeatletter
        \newbox\Wrappedcontinuationbox 
        \newbox\Wrappedvisiblespacebox 
        \newcommand*\Wrappedvisiblespace {\textcolor{red}{\textvisiblespace}} 
        \newcommand*\Wrappedcontinuationsymbol {\textcolor{red}{\llap{\tiny$\m@th\hookrightarrow$}}} 
        \newcommand*\Wrappedcontinuationindent {3ex } 
        \newcommand*\Wrappedafterbreak {\kern\Wrappedcontinuationindent\copy\Wrappedcontinuationbox} 
        % Take advantage of the already applied Pygments mark-up to insert 
        % potential linebreaks for TeX processing. 
        %        {, <, #, %, $, ' and ": go to next line. 
        %        _, }, ^, &, >, - and ~: stay at end of broken line. 
        % Use of \textquotesingle for straight quote. 
        \newcommand*\Wrappedbreaksatspecials {% 
            \def\PYGZus{\discretionary{\char`\_}{\Wrappedafterbreak}{\char`\_}}% 
            \def\PYGZob{\discretionary{}{\Wrappedafterbreak\char`\{}{\char`\{}}% 
            \def\PYGZcb{\discretionary{\char`\}}{\Wrappedafterbreak}{\char`\}}}% 
            \def\PYGZca{\discretionary{\char`\^}{\Wrappedafterbreak}{\char`\^}}% 
            \def\PYGZam{\discretionary{\char`\&}{\Wrappedafterbreak}{\char`\&}}% 
            \def\PYGZlt{\discretionary{}{\Wrappedafterbreak\char`\<}{\char`\<}}% 
            \def\PYGZgt{\discretionary{\char`\>}{\Wrappedafterbreak}{\char`\>}}% 
            \def\PYGZsh{\discretionary{}{\Wrappedafterbreak\char`\#}{\char`\#}}% 
            \def\PYGZpc{\discretionary{}{\Wrappedafterbreak\char`\%}{\char`\%}}% 
            \def\PYGZdl{\discretionary{}{\Wrappedafterbreak\char`\$}{\char`\$}}% 
            \def\PYGZhy{\discretionary{\char`\-}{\Wrappedafterbreak}{\char`\-}}% 
            \def\PYGZsq{\discretionary{}{\Wrappedafterbreak\textquotesingle}{\textquotesingle}}% 
            \def\PYGZdq{\discretionary{}{\Wrappedafterbreak\char`\"}{\char`\"}}% 
            \def\PYGZti{\discretionary{\char`\~}{\Wrappedafterbreak}{\char`\~}}% 
        } 
        % Some characters . , ; ? ! / are not pygmentized. 
        % This macro makes them "active" and they will insert potential linebreaks 
        \newcommand*\Wrappedbreaksatpunct {% 
            \lccode`\~`\.\lowercase{\def~}{\discretionary{\hbox{\char`\.}}{\Wrappedafterbreak}{\hbox{\char`\.}}}% 
            \lccode`\~`\,\lowercase{\def~}{\discretionary{\hbox{\char`\,}}{\Wrappedafterbreak}{\hbox{\char`\,}}}% 
            \lccode`\~`\;\lowercase{\def~}{\discretionary{\hbox{\char`\;}}{\Wrappedafterbreak}{\hbox{\char`\;}}}% 
            \lccode`\~`\:\lowercase{\def~}{\discretionary{\hbox{\char`\:}}{\Wrappedafterbreak}{\hbox{\char`\:}}}% 
            \lccode`\~`\?\lowercase{\def~}{\discretionary{\hbox{\char`\?}}{\Wrappedafterbreak}{\hbox{\char`\?}}}% 
            \lccode`\~`\!\lowercase{\def~}{\discretionary{\hbox{\char`\!}}{\Wrappedafterbreak}{\hbox{\char`\!}}}% 
            \lccode`\~`\/\lowercase{\def~}{\discretionary{\hbox{\char`\/}}{\Wrappedafterbreak}{\hbox{\char`\/}}}% 
            \catcode`\.\active
            \catcode`\,\active 
            \catcode`\;\active
            \catcode`\:\active
            \catcode`\?\active
            \catcode`\!\active
            \catcode`\/\active 
            \lccode`\~`\~ 	
        }
    \makeatother

    \let\OriginalVerbatim=\Verbatim
    \makeatletter
    \renewcommand{\Verbatim}[1][1]{%
        %\parskip\z@skip
        \sbox\Wrappedcontinuationbox {\Wrappedcontinuationsymbol}%
        \sbox\Wrappedvisiblespacebox {\FV@SetupFont\Wrappedvisiblespace}%
        \def\FancyVerbFormatLine ##1{\hsize\linewidth
            \vtop{\raggedright\hyphenpenalty\z@\exhyphenpenalty\z@
                \doublehyphendemerits\z@\finalhyphendemerits\z@
                \strut ##1\strut}%
        }%
        % If the linebreak is at a space, the latter will be displayed as visible
        % space at end of first line, and a continuation symbol starts next line.
        % Stretch/shrink are however usually zero for typewriter font.
        \def\FV@Space {%
            \nobreak\hskip\z@ plus\fontdimen3\font minus\fontdimen4\font
            \discretionary{\copy\Wrappedvisiblespacebox}{\Wrappedafterbreak}
            {\kern\fontdimen2\font}%
        }%
        
        % Allow breaks at special characters using \PYG... macros.
        \Wrappedbreaksatspecials
        % Breaks at punctuation characters . , ; ? ! and / need catcode=\active 	
        \OriginalVerbatim[#1,codes*=\Wrappedbreaksatpunct]%
    }
    \makeatother

    % Exact colors from NB
    \definecolor{incolor}{HTML}{303F9F}
    \definecolor{outcolor}{HTML}{D84315}
    \definecolor{cellborder}{HTML}{CFCFCF}
    \definecolor{cellbackground}{HTML}{F7F7F7}
    
    % prompt
    \makeatletter
    \newcommand{\boxspacing}{\kern\kvtcb@left@rule\kern\kvtcb@boxsep}
    \makeatother
    \newcommand{\prompt}[4]{
        {\ttfamily\llap{{\color{#2}[#3]:\hspace{3pt}#4}}\vspace{-\baselineskip}}
    }
    

    
    % Prevent overflowing lines due to hard-to-break entities
    \sloppy 
    % Setup hyperref package
    \hypersetup{
      breaklinks=true,  % so long urls are correctly broken across lines
      colorlinks=true,
      urlcolor=urlcolor,
      linkcolor=linkcolor,
      citecolor=citecolor,
      }
    % Slightly bigger margins than the latex defaults
    
    \geometry{verbose,tmargin=1in,bmargin=1in,lmargin=1in,rmargin=1in}
    
    

\begin{document}
    
    \maketitle
    
    

    
    \hypertarget{mads-notebook-contamination-problem}{%
\section{Mads Notebook: Contamination
Problem}\label{mads-notebook-contamination-problem}}

\href{http://madsjulia.github.io/Mads.jl}{MADS} is an integrated
high-performance computational framework for data/model/decision
analyses.

\begin{verbatim}
<img src="../../logo/mads_black_swan_logo_big_text_new_3inch.png" alt="MADS" width=20% max-width=125px;/>
\end{verbatim}

\href{http://madsjulia.github.io/Mads.jl}{MADS} can be applied to
perform:

\begin{itemize}
\tightlist
\item
  Sensitivity Analysis
\item
  Parameter Estimation
\item
  Model Inversion and Calibration
\item
  Uncertainty Quantification
\item
  Model Selection and Model Averaging
\item
  Model Reduction and Surrogate Modeling
\item
  Machine Learning (e.g., Blind Source Separation, Source
  Identification, Feature Extraction, Matrix / Tensor Factorization,
  etc.)
\item
  Decision Analysis and Support
\end{itemize}

Here, it is demonstrated how
\href{http://madsjulia.github.io/Mads.jl}{MADS} can be applied to solve
a general groundwater contamination problem.

\href{http://madsjulia.github.io/Mads.jl}{MADS} includes analytical
solver called \href{http://madsjulia.github.io/Anasol.jl}{Anasol.jl}
which is appied to solve the groundwater contamination transport in a
aquifer as presetned below.

\hypertarget{problem-setup}{%
\subsection{Problem setup}\label{problem-setup}}

Import Mads (if \textbf{MADS} is not installed, first execute in the
Julia REPL: \texttt{import\ Pkg;\ Pkg.add("Mads")}):

    \begin{tcolorbox}[breakable, size=fbox, boxrule=1pt, pad at break*=1mm,colback=cellbackground, colframe=cellborder]
\prompt{In}{incolor}{1}{\boxspacing}
\begin{Verbatim}[commandchars=\\\{\}]
\PY{k}{import} \PY{n}{Pkg}\PY{p}{;} \PY{n}{Pkg}\PY{o}{.}\PY{n}{resolve}\PY{p}{(}\PY{p}{)}
\PY{k}{import} \PY{n}{Revise}
\PY{k}{import} \PY{n}{Mads}
\end{Verbatim}
\end{tcolorbox}

    \begin{Verbatim}[commandchars=\\\{\}]
┌ Info: Precompiling Mads [d6bdc55b-bd94-5012-933c-1f73fc2ee992]
└ @ Base loading.jl:1317
    \end{Verbatim}

    \begin{Verbatim}[commandchars=\\\{\}]
\textbf{Mads: Model Analysis \& Decision Support}
====

\textcolor{ansi-blue-intense}{\textbf{    \_\_\_      \_\_\_\_    }}\textcolor{ansi-red-intense}{\textbf{        \_\_\_\_   }}\textcolor{ansi-green-intense}{\textbf{ \_\_\_\_
}}\textcolor{ansi-magenta-intense}{\textbf{     \_\_\_\_\_\_}}
\textcolor{ansi-blue-intense}{\textbf{   /   \textbackslash{}    /    \textbackslash{}  }}\textcolor{ansi-red-intense}{\textbf{        /    | }}\textcolor{ansi-green-intense}{\textbf{ |    \textbackslash{}
}}\textcolor{ansi-magenta-intense}{\textbf{       /  \_\_  \textbackslash{}}}
\textcolor{ansi-blue-intense}{\textbf{  |     \textbackslash{}  /     |   }}\textcolor{ansi-red-intense}{\textbf{      /     |  }}\textcolor{ansi-green-intense}{\textbf{|     \textbackslash{}
}}\textcolor{ansi-magenta-intense}{\textbf{     /  /  \textbackslash{}\_\_\textbackslash{}}}
\textcolor{ansi-blue-intense}{\textbf{  |  |\textbackslash{}  \textbackslash{}/  /|  | }}\textcolor{ansi-red-intense}{\textbf{       /      | }}\textcolor{ansi-green-intense}{\textbf{ |      \textbackslash{}
}}\textcolor{ansi-magenta-intense}{\textbf{     |  |}}
\textcolor{ansi-blue-intense}{\textbf{  |  | \textbackslash{}    / |  |  }}\textcolor{ansi-red-intense}{\textbf{     /  /|   | }}\textcolor{ansi-green-intense}{\textbf{ |   |\textbackslash{}  \textbackslash{}
}}\textcolor{ansi-magenta-intense}{\textbf{     \textbackslash{}  \textbackslash{}\_\_\_\_\_\_.}}
\textcolor{ansi-blue-intense}{\textbf{  |  |  \textbackslash{}\_\_/  |  |  }}\textcolor{ansi-red-intense}{\textbf{    /  / |   | }}\textcolor{ansi-green-intense}{\textbf{ |   | \textbackslash{}  \textbackslash{}
}}\textcolor{ansi-magenta-intense}{\textbf{      \textbackslash{}\_\_\_\_\_\_\_  \textbackslash{}}}
\textcolor{ansi-blue-intense}{\textbf{  |  |        |  | }}\textcolor{ansi-red-intense}{\textbf{    /  /  |   | }}\textcolor{ansi-green-intense}{\textbf{ |   |  \textbackslash{}  \textbackslash{}
}}\textcolor{ansi-magenta-intense}{\textbf{             \textbackslash{}  \textbackslash{}}}
\textcolor{ansi-blue-intense}{\textbf{  |  |        |  |  }}\textcolor{ansi-red-intense}{\textbf{  /  /===|   | }}\textcolor{ansi-green-intense}{\textbf{ |   |\_\_\_\textbackslash{}  \textbackslash{}
}}\textcolor{ansi-magenta-intense}{\textbf{   \_\_.        |  |}}
\textcolor{ansi-blue-intense}{\textbf{  |  |        |  | }}\textcolor{ansi-red-intense}{\textbf{  /  /    |   | }}\textcolor{ansi-green-intense}{\textbf{ |           \textbackslash{}
}}\textcolor{ansi-magenta-intense}{\textbf{ \textbackslash{}  \textbackslash{}\_\_\_\_\_\_/  /}}
\textcolor{ansi-blue-intense}{\textbf{  |\_\_|        |\_\_| }}\textcolor{ansi-red-intense}{\textbf{ /\_\_/     |\_\_\_| }}\textcolor{ansi-green-intense}{\textbf{ |\_\_\_\_\_\_\_\_\_\_\_\_\textbackslash{}
}}\textcolor{ansi-magenta-intense}{\textbf{  \textbackslash{}\_\_\_\_\_\_\_\_\_\_/}}

\textbf{MADS} is an integrated high-performance computational framework for data-
and model-based analyses.
\textbf{MADS} can perform: Sensitivity Analysis, Parameter Estimation, Model
Inversion and Calibration, Uncertainty Quantification, Model Selection and Model
Averaging, Model Reduction and Surrogate Modeling, Machine Learning, Decision
Analysis and Support.
    \end{Verbatim}

    \begin{Verbatim}[commandchars=\\\{\}]
WARNING: Method definition getparamsmin(Base.AbstractDict\{K, V\} where V where K)
in module Mads at /Users/vvv/.julia/dev/Mads/src/MadsParameters.jl:148
overwritten at /Users/vvv/.julia/dev/Mads/src/MadsParameters.jl:170.
  ** incremental compilation may be fatally broken for this module **

\textcolor{ansi-green-intense}{\textbf{    Updating}} registry at `\textasciitilde{}/.julia/registries/General`
\textcolor{ansi-yellow-intense}{\textbf{┌ }}\textcolor{ansi-yellow-intense}{\textbf{Warning: }}could not download
https://pkg.julialang.org/registries
\textcolor{ansi-yellow-intense}{\textbf{└ }}\textcolor{ansi-black-intense}{@ Pkg.Types /Users/julia/buildbot/worker/package\_macos
64/build/usr/share/julia/stdlib/v1.6/Pkg/src/Types.jl:980}
\textcolor{ansi-green-intense}{\textbf{   Resolving}} package versions{\ldots}
\textcolor{ansi-cyan-intense}{\textbf{[ }}\textcolor{ansi-cyan-intense}{\textbf{Info: }}Module BIGUQ is not available!
┌ Info: Installing pyqt package to avoid buggy tkagg backend.
└ @ PyPlot /Users/vvv/.julia/packages/PyPlot/XHEG0/src/init.jl:118
    \end{Verbatim}

    Change the working directory

    \begin{tcolorbox}[breakable, size=fbox, boxrule=1pt, pad at break*=1mm,colback=cellbackground, colframe=cellborder]
\prompt{In}{incolor}{2}{\boxspacing}
\begin{Verbatim}[commandchars=\\\{\}]
\PY{n}{cd}\PY{p}{(}\PY{n}{joinpath}\PY{p}{(}\PY{n}{Mads}\PY{o}{.}\PY{n}{madsdir}\PY{p}{,} \PY{l+s}{\PYZdq{}}\PY{l+s}{examples}\PY{l+s}{\PYZdq{}}\PY{p}{,} \PY{l+s}{\PYZdq{}}\PY{l+s}{contamination}\PY{l+s}{\PYZdq{}}\PY{p}{)}\PY{p}{)}
\end{Verbatim}
\end{tcolorbox}

    Load Mads input file

    \begin{tcolorbox}[breakable, size=fbox, boxrule=1pt, pad at break*=1mm,colback=cellbackground, colframe=cellborder]
\prompt{In}{incolor}{3}{\boxspacing}
\begin{Verbatim}[commandchars=\\\{\}]
\PY{n}{md} \PY{o}{=} \PY{n}{Mads}\PY{o}{.}\PY{n}{loadmadsfile}\PY{p}{(}\PY{l+s}{\PYZdq{}}\PY{l+s}{w01.mads}\PY{l+s}{\PYZdq{}}\PY{p}{)}
\end{Verbatim}
\end{tcolorbox}

            \begin{tcolorbox}[breakable, size=fbox, boxrule=.5pt, pad at break*=1mm, opacityfill=0]
\prompt{Out}{outcolor}{3}{\boxspacing}
\begin{Verbatim}[commandchars=\\\{\}]
Dict\{String, Any\} with 7 entries:
  "Grid"         => Dict\{Any, Any\}("zmax"=>50, "time"=>50, "xcount"=>33, "zcoun…
  "Sources"      => Dict\{Any, Any\}[Dict("box"=>Dict\{Any, Any\}("dz"=>Dict\{Any, A…
  "Parameters"   => OrderedCollections.OrderedDict\{String, OrderedCollections.O…
  "Wells"        => OrderedCollections.OrderedDict\{String, Any\}("w1a"=>Dict\{Any…
  "Time"         => Dict\{Any, Any\}("step"=>1, "start"=>1, "end"=>50)
  "Observations" => OrderedCollections.OrderedDict\{Any, Any\}("w1a\_1"=>OrderedCo…
  "Filename"     => "w01.mads"
\end{Verbatim}
\end{tcolorbox}
        
    Generate a plot of the loaded problem showing the well locations and the
location of the contaminant source:

    \begin{tcolorbox}[breakable, size=fbox, boxrule=1pt, pad at break*=1mm,colback=cellbackground, colframe=cellborder]
\prompt{In}{incolor}{4}{\boxspacing}
\begin{Verbatim}[commandchars=\\\{\}]
\PY{n}{Mads}\PY{o}{.}\PY{n}{plotmadsproblem}\PY{p}{(}\PY{n}{md}\PY{p}{,} \PY{n}{keyword}\PY{o}{=}\PY{l+s}{\PYZdq{}}\PY{l+s}{all\PYZus{}wells}\PY{l+s}{\PYZdq{}}\PY{p}{)}
\end{Verbatim}
\end{tcolorbox}

    \begin{center}
    \adjustimage{max size={0.9\linewidth}{0.9\paperheight}}{contamination_files/contamination_7_0.png}
    \end{center}
    { \hspace*{\fill} \\}
    
    \begin{Verbatim}[commandchars=\\\{\}]

    \end{Verbatim}

    There are 20 monitoring wells. Each well has 2 measurement ports:
shallow (3 m below the water table labeled \texttt{a}) and deep (33 m
below the water table labeled \texttt{b}). Contaminant concentrations
are observed for 50 years at each well. The contaminant transport is
solved using the \texttt{Anasol} package in Mads.

    \hypertarget{unknown-model-parameters}{%
\subsubsection{Unknown model
parameters}\label{unknown-model-parameters}}

\begin{itemize}
\tightlist
\item
  Start time of contaminant release \(t_0\)
\item
  End time of contaminant release \(t_1\)
\item
  Advective pore velocity \(v\)
\end{itemize}

    \hypertarget{reduced-model-setup}{%
\subsubsection{Reduced model setup}\label{reduced-model-setup}}

Analysis of the data from only 2 monitoring locations: \texttt{w13a} and
\texttt{w20a}.

    \begin{tcolorbox}[breakable, size=fbox, boxrule=1pt, pad at break*=1mm,colback=cellbackground, colframe=cellborder]
\prompt{In}{incolor}{5}{\boxspacing}
\begin{Verbatim}[commandchars=\\\{\}]
\PY{n}{Mads}\PY{o}{.}\PY{n}{allwellsoff!}\PY{p}{(}\PY{n}{md}\PY{p}{)} \PY{c}{\PYZsh{} turn off all wells}
\PY{n}{Mads}\PY{o}{.}\PY{n}{wellon!}\PY{p}{(}\PY{n}{md}\PY{p}{,} \PY{l+s}{\PYZdq{}}\PY{l+s}{w13a}\PY{l+s}{\PYZdq{}}\PY{p}{)} \PY{c}{\PYZsh{} use well w13a}
\PY{n}{Mads}\PY{o}{.}\PY{n}{wellon!}\PY{p}{(}\PY{n}{md}\PY{p}{,} \PY{l+s}{\PYZdq{}}\PY{l+s}{w20a}\PY{l+s}{\PYZdq{}}\PY{p}{)} \PY{c}{\PYZsh{} use well w20a}
\end{Verbatim}
\end{tcolorbox}

            \begin{tcolorbox}[breakable, size=fbox, boxrule=.5pt, pad at break*=1mm, opacityfill=0]
\prompt{Out}{outcolor}{5}{\boxspacing}
\begin{Verbatim}[commandchars=\\\{\}]
OrderedCollections.OrderedDict\{Any, Any\} with 100 entries:
  "w13a\_1"  => OrderedCollections.OrderedDict\{Any, Any\}("well"=>"w13a", "time"=…
  "w13a\_2"  => OrderedCollections.OrderedDict\{Any, Any\}("well"=>"w13a", "time"=…
  "w13a\_3"  => OrderedCollections.OrderedDict\{Any, Any\}("well"=>"w13a", "time"=…
  "w13a\_4"  => OrderedCollections.OrderedDict\{Any, Any\}("well"=>"w13a", "time"=…
  "w13a\_5"  => OrderedCollections.OrderedDict\{Any, Any\}("well"=>"w13a", "time"=…
  "w13a\_6"  => OrderedCollections.OrderedDict\{Any, Any\}("well"=>"w13a", "time"=…
  "w13a\_7"  => OrderedCollections.OrderedDict\{Any, Any\}("well"=>"w13a", "time"=…
  "w13a\_8"  => OrderedCollections.OrderedDict\{Any, Any\}("well"=>"w13a", "time"=…
  "w13a\_9"  => OrderedCollections.OrderedDict\{Any, Any\}("well"=>"w13a", "time"=…
  "w13a\_10" => OrderedCollections.OrderedDict\{Any, Any\}("well"=>"w13a", "time"=…
  "w13a\_11" => OrderedCollections.OrderedDict\{Any, Any\}("well"=>"w13a", "time"=…
  "w13a\_12" => OrderedCollections.OrderedDict\{Any, Any\}("well"=>"w13a", "time"=…
  "w13a\_13" => OrderedCollections.OrderedDict\{Any, Any\}("well"=>"w13a", "time"=…
  "w13a\_14" => OrderedCollections.OrderedDict\{Any, Any\}("well"=>"w13a", "time"=…
  "w13a\_15" => OrderedCollections.OrderedDict\{Any, Any\}("well"=>"w13a", "time"=…
  "w13a\_16" => OrderedCollections.OrderedDict\{Any, Any\}("well"=>"w13a", "time"=…
  "w13a\_17" => OrderedCollections.OrderedDict\{Any, Any\}("well"=>"w13a", "time"=…
  "w13a\_18" => OrderedCollections.OrderedDict\{Any, Any\}("well"=>"w13a", "time"=…
  "w13a\_19" => OrderedCollections.OrderedDict\{Any, Any\}("well"=>"w13a", "time"=…
  "w13a\_20" => OrderedCollections.OrderedDict\{Any, Any\}("well"=>"w13a", "time"=…
  "w13a\_21" => OrderedCollections.OrderedDict\{Any, Any\}("well"=>"w13a", "time"=…
  "w13a\_22" => OrderedCollections.OrderedDict\{Any, Any\}("well"=>"w13a", "time"=…
  "w13a\_23" => OrderedCollections.OrderedDict\{Any, Any\}("well"=>"w13a", "time"=…
  "w13a\_24" => OrderedCollections.OrderedDict\{Any, Any\}("well"=>"w13a", "time"=…
  "w13a\_25" => OrderedCollections.OrderedDict\{Any, Any\}("well"=>"w13a", "time"=…
  ⋮         => ⋮
\end{Verbatim}
\end{tcolorbox}
        
    Generate a plot of the updated problem showing the 2 well locations
applied in the analyses as well as the location of the contaminant
source:

    \begin{tcolorbox}[breakable, size=fbox, boxrule=1pt, pad at break*=1mm,colback=cellbackground, colframe=cellborder]
\prompt{In}{incolor}{6}{\boxspacing}
\begin{Verbatim}[commandchars=\\\{\}]
\PY{n}{Mads}\PY{o}{.}\PY{n}{plotmadsproblem}\PY{p}{(}\PY{n}{md}\PY{p}{;} \PY{n}{keyword}\PY{o}{=}\PY{l+s}{\PYZdq{}}\PY{l+s}{w13a\PYZus{}w20a}\PY{l+s}{\PYZdq{}}\PY{p}{)}
\end{Verbatim}
\end{tcolorbox}

    \begin{center}
    \adjustimage{max size={0.9\linewidth}{0.9\paperheight}}{contamination_files/contamination_13_0.png}
    \end{center}
    { \hspace*{\fill} \\}
    
    \begin{Verbatim}[commandchars=\\\{\}]

    \end{Verbatim}

    \hypertarget{initial-estimates}{%
\subsection{Initial estimates}\label{initial-estimates}}

    Plot initial estimates of the contamiant concentrations at the 2
monitoring wells based on the initial model parameters:

    \begin{tcolorbox}[breakable, size=fbox, boxrule=1pt, pad at break*=1mm,colback=cellbackground, colframe=cellborder]
\prompt{In}{incolor}{7}{\boxspacing}
\begin{Verbatim}[commandchars=\\\{\}]
\PY{n}{Mads}\PY{o}{.}\PY{n}{plotmatches}\PY{p}{(}\PY{n}{md}\PY{p}{,} \PY{l+s}{\PYZdq{}}\PY{l+s}{w13a}\PY{l+s}{\PYZdq{}}\PY{p}{;} \PY{n}{display}\PY{o}{=}\PY{n+nb}{true}\PY{p}{)}
\end{Verbatim}
\end{tcolorbox}

    \begin{center}
    \adjustimage{max size={0.9\linewidth}{0.9\paperheight}}{contamination_files/contamination_16_0.png}
    \end{center}
    { \hspace*{\fill} \\}
    
    \begin{Verbatim}[commandchars=\\\{\}]

    \end{Verbatim}

    \begin{tcolorbox}[breakable, size=fbox, boxrule=1pt, pad at break*=1mm,colback=cellbackground, colframe=cellborder]
\prompt{In}{incolor}{8}{\boxspacing}
\begin{Verbatim}[commandchars=\\\{\}]
\PY{n}{Mads}\PY{o}{.}\PY{n}{plotmatches}\PY{p}{(}\PY{n}{md}\PY{p}{,} \PY{l+s}{\PYZdq{}}\PY{l+s}{w20a}\PY{l+s}{\PYZdq{}}\PY{p}{;} \PY{n}{display}\PY{o}{=}\PY{n+nb}{true}\PY{p}{)}
\end{Verbatim}
\end{tcolorbox}

    \begin{center}
    \adjustimage{max size={0.9\linewidth}{0.9\paperheight}}{contamination_files/contamination_17_0.png}
    \end{center}
    { \hspace*{\fill} \\}
    
    \begin{Verbatim}[commandchars=\\\{\}]

    \end{Verbatim}

    \hypertarget{model-calibration}{%
\subsection{Model calibration}\label{model-calibration}}

    Execute model calibration based on the concentrations observed in the
two monitoring wells:

    \begin{tcolorbox}[breakable, size=fbox, boxrule=1pt, pad at break*=1mm,colback=cellbackground, colframe=cellborder]
\prompt{In}{incolor}{9}{\boxspacing}
\begin{Verbatim}[commandchars=\\\{\}]
\PY{n}{calib\PYZus{}param}\PY{p}{,} \PY{n}{calib\PYZus{}results} \PY{o}{=} \PY{n}{Mads}\PY{o}{.}\PY{n}{calibrate}\PY{p}{(}\PY{n}{md}\PY{p}{)}
\end{Verbatim}
\end{tcolorbox}

            \begin{tcolorbox}[breakable, size=fbox, boxrule=.5pt, pad at break*=1mm, opacityfill=0]
\prompt{Out}{outcolor}{9}{\boxspacing}
\begin{Verbatim}[commandchars=\\\{\}]
(OrderedCollections.OrderedDict("n" => 0.1, "rf" => 1.0, "lambda" => 0.0,
"theta" => 0.0, "vx" => 31.059669248076222, "vy" => 0.0, "vz" => 0.0, "ax" =>
70.0, "ay" => 15.0, "az" => 0.3…),
OptimBase.MultivariateOptimizationResults\{LsqFit.LevenbergMarquardt, Float64,
1\}(LsqFit.LevenbergMarquardt(), [0.740931532960472, -0.20135792079033074,
-0.44291104407363896], [0.6737086338839451, 0.007322888567510982,
-0.34261232076929443], 41650.46179056277, 13, false, true, 0.0001, 0.0, false,
0.001, 0.0, false, 1.0e-6, 0.0, false, Iter     Function value   Gradient norm
------   --------------   --------------
, 170, 13, 0))
\end{Verbatim}
\end{tcolorbox}
        
    Compute forward model predictions based on the calibrated model
parameters:

    \begin{tcolorbox}[breakable, size=fbox, boxrule=1pt, pad at break*=1mm,colback=cellbackground, colframe=cellborder]
\prompt{In}{incolor}{10}{\boxspacing}
\begin{Verbatim}[commandchars=\\\{\}]
\PY{n}{calib\PYZus{}predictions} \PY{o}{=} \PY{n}{Mads}\PY{o}{.}\PY{n}{forward}\PY{p}{(}\PY{n}{md}\PY{p}{,} \PY{n}{calib\PYZus{}param}\PY{p}{)}
\end{Verbatim}
\end{tcolorbox}

            \begin{tcolorbox}[breakable, size=fbox, boxrule=.5pt, pad at break*=1mm, opacityfill=0]
\prompt{Out}{outcolor}{10}{\boxspacing}
\begin{Verbatim}[commandchars=\\\{\}]
OrderedCollections.OrderedDict\{Any, Float64\} with 100 entries:
  "w13a\_1"  => 0.0
  "w13a\_2"  => 0.0
  "w13a\_3"  => 0.0
  "w13a\_4"  => 0.0
  "w13a\_5"  => 0.0
  "w13a\_6"  => 4.79956e-11
  "w13a\_7"  => 0.000284228
  "w13a\_8"  => 0.0590933
  "w13a\_9"  => 0.92868
  "w13a\_10" => 5.08796
  "w13a\_11" => 16.2469
  "w13a\_12" => 37.7882
  "w13a\_13" => 71.6886
  "w13a\_14" => 118.275
  "w13a\_15" => 176.509
  "w13a\_16" => 244.465
  "w13a\_17" => 319.785
  "w13a\_18" => 400.036
  "w13a\_19" => 482.934
  "w13a\_20" => 566.28
  "w13a\_21" => 647.03
  "w13a\_22" => 720.732
  "w13a\_23" => 782.658
  "w13a\_24" => 829.249
  "w13a\_25" => 858.781
  ⋮         => ⋮
\end{Verbatim}
\end{tcolorbox}
        
    Plot the predicted estimates of the contamiant concentrations at the 2
monitoring wells based on the estimated model parameters based on the
performed model calibration:

    \begin{tcolorbox}[breakable, size=fbox, boxrule=1pt, pad at break*=1mm,colback=cellbackground, colframe=cellborder]
\prompt{In}{incolor}{11}{\boxspacing}
\begin{Verbatim}[commandchars=\\\{\}]
\PY{n}{Mads}\PY{o}{.}\PY{n}{plotmatches}\PY{p}{(}\PY{n}{md}\PY{p}{,} \PY{n}{calib\PYZus{}predictions}\PY{p}{,} \PY{l+s}{\PYZdq{}}\PY{l+s}{w13a}\PY{l+s}{\PYZdq{}}\PY{p}{)}
\end{Verbatim}
\end{tcolorbox}

    \begin{center}
    \adjustimage{max size={0.9\linewidth}{0.9\paperheight}}{contamination_files/contamination_24_0.png}
    \end{center}
    { \hspace*{\fill} \\}
    
    \begin{Verbatim}[commandchars=\\\{\}]

    \end{Verbatim}

    \begin{tcolorbox}[breakable, size=fbox, boxrule=1pt, pad at break*=1mm,colback=cellbackground, colframe=cellborder]
\prompt{In}{incolor}{12}{\boxspacing}
\begin{Verbatim}[commandchars=\\\{\}]
\PY{n}{Mads}\PY{o}{.}\PY{n}{plotmatches}\PY{p}{(}\PY{n}{md}\PY{p}{,} \PY{n}{calib\PYZus{}predictions}\PY{p}{,} \PY{l+s}{\PYZdq{}}\PY{l+s}{w20a}\PY{l+s}{\PYZdq{}}\PY{p}{)}
\end{Verbatim}
\end{tcolorbox}

    \begin{center}
    \adjustimage{max size={0.9\linewidth}{0.9\paperheight}}{contamination_files/contamination_25_0.png}
    \end{center}
    { \hspace*{\fill} \\}
    
    \begin{Verbatim}[commandchars=\\\{\}]

    \end{Verbatim}

    Initial values of the optimized model parameters are:

    \begin{tcolorbox}[breakable, size=fbox, boxrule=1pt, pad at break*=1mm,colback=cellbackground, colframe=cellborder]
\prompt{In}{incolor}{13}{\boxspacing}
\begin{Verbatim}[commandchars=\\\{\}]
\PY{n}{Mads}\PY{o}{.}\PY{n}{showparameters}\PY{p}{(}\PY{n}{md}\PY{p}{)}
\end{Verbatim}
\end{tcolorbox}

    \begin{Verbatim}[commandchars=\\\{\}]
Pore x velocity [L/T] : vx       =              40 log-transformed min = 0.01
max = 200.0
Start Time [T]        : source1\_t0 =               4 min = 0.0 max = 10.0
End Time [T]          : source1\_t1 =              15 min = 5.0 max = 40.0
Number of optimizable parameters: 3
    \end{Verbatim}

    Estimated values of the optimized model parameters are:

    \begin{tcolorbox}[breakable, size=fbox, boxrule=1pt, pad at break*=1mm,colback=cellbackground, colframe=cellborder]
\prompt{In}{incolor}{14}{\boxspacing}
\begin{Verbatim}[commandchars=\\\{\}]
\PY{n}{Mads}\PY{o}{.}\PY{n}{showparameterestimates}\PY{p}{(}\PY{n}{md}\PY{p}{,} \PY{n}{calib\PYZus{}param}\PY{p}{)}
\end{Verbatim}
\end{tcolorbox}

            \begin{tcolorbox}[breakable, size=fbox, boxrule=.5pt, pad at break*=1mm, opacityfill=0]
\prompt{Out}{outcolor}{14}{\boxspacing}
\begin{Verbatim}[commandchars=\\\{\}]
3-element Vector\{Pair\{String, Float64\}\}:
         "vx" => 31.059669248076222
 "source1\_t0" => 5.036614115598699
 "source1\_t1" => 16.62089724181972
\end{Verbatim}
\end{tcolorbox}
        
    \begin{tcolorbox}[breakable, size=fbox, boxrule=1pt, pad at break*=1mm,colback=cellbackground, colframe=cellborder]
\prompt{In}{incolor}{ }{\boxspacing}
\begin{Verbatim}[commandchars=\\\{\}]

\end{Verbatim}
\end{tcolorbox}


    % Add a bibliography block to the postdoc
    
    
    
\end{document}
